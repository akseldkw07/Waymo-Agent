\documentclass[12pt]{article}

\newcommand{\duedate}{10/09/2025}
\newcommand{\assignment}{Architecture 10/09} % Change to "Problem Set X"

% Change the following to your name and UNI.
\newcommand{\name}{Aksel Kretsinger-Walters, adk2164}
\newcommand{\email}{adk2164@columbia.edu}

% NOTE: Defining collaborators is optional; to not list collaborators, comment out the
% line below. Maximum of two collaborators per problem set.
%\newcommand{\collaborators}{Vig Vigerton (\texttt{UNI}), Alice Bob (\texttt{UNI})}
% No collaborators on PS 0

\makeatletter
\def\input@path{{../}{../../}{../../../}} % add as many parents as you need
\makeatother
% Copyright 2021 Paolo Adajar (padajar.com, paoloadajar@mit.edu)
%
% Permission is hereby granted, free of charge, to any person obtaining a copy of this
% software and associated documentation files (the "Software"), to deal in the Software
% without restriction, including without limitation the rights to use, copy, modify,
% merge, publish, distribute, sublicense, and/or sell copies of the Software, and to
% permit persons to whom the Software is furnished to do so, subject to the following conditions:
%
% The above copyright notice and this permission notice shall be included in all copies
% or substantial portions of the Software.
%
% THE SOFTWARE IS PROVIDED "AS IS", WITHOUT WARRANTY OF ANY KIND, EXPRESS OR IMPLIED,
% INCLUDING BUT NOT LIMITED TO THE WARRANTIES OF MERCHANTABILITY, FITNESS FOR A
% PARTICULAR PURPOSE AND NONINFRINGEMENT. IN NO EVENT SHALL THE AUTHORS OR COPYRIGHT
% HOLDERS BE LIABLE FOR ANY CLAIM, DAMAGES OR OTHER LIABILITY, WHETHER IN AN ACTION OF
% CONTRACT, TORT OR OTHERWISE, ARISING FROM, OUT OF OR IN CONNECTION WITH THE SOFTWARE OR
% THE USE OR OTHER DEALINGS IN THE SOFTWARE.

%%%%%%%%%%%%%%%%%%%%%%%%%%%%%%%%%%%%%%
%%%%% DO NOT MODIFY THIS FILE %%%%%%%%
%%%%%%%%%%%%%%%%%%%%%%%%%%%%%%%%%%%%%%

%%%%%%%%%%%%%%%%%%%%%%%%%%%%%%
%%%%% CLASS SPECIFICS %%%%%%%%
%%%%%%%%%%%%%%%%%%%%%%%%%%%%%%
\newcommand{\classnum}{ORCS 4529}
\newcommand{\subject}{Reinforcement Learning}
\newcommand{\instructors}{Shipra Agrawal}
\newcommand{\semester}{Fall 2025}

%%%%%%%%%%%%%%%%%%%%%%%%%%%%%%
%%%%% PACKAGE IMPORTS %%%%%%%%
%%%%%%%%%%%%%%%%%%%%%%%%%%%%%%
\usepackage{fullpage}
\usepackage{enumitem}
\usepackage{amsfonts, amssymb, amsmath,amsthm}
\usepackage{tikz}
\usepackage{hyperref}
\usepackage{ifthen}

\hypersetup{
		colorlinks=true,
		linkcolor=blue,
		filecolor=magenta,
		urlcolor=blue,
}

%%%%%%%%%%%%%%%%%%%%%%%%%%%%
%%%%% CUSTOM MACROS %%%%%%%%
%%%%%%%%%%%%%%%%%%%%%%%%%%%%
\usepackage{macros}

%%%%%%%%%%%%%%%%%%%%%%%%%
%%%%% FORMATTING %%%%%%%%
%%%%%%%%%%%%%%%%%%%%%%%%%
\setlength{\parindent}{0mm}
\setlength{\parskip}{2mm}

\setlist[enumerate]{label=({\alph*})}
\setlist[enumerate, 2]{label=({\roman*})}

\allowdisplaybreaks[1]

%%%%%%%%%%%%%%%%%%%%%
%%%%% HEADER %%%%%%%%
%%%%%%%%%%%%%%%%%%%%%
\newcommand{\psetheader}{
		\ifthenelse{\isundefined{\collaborators}}{
				\begin{center}
						{\setlength{\parindent}{0cm} \setlength{\parskip}{0mm}

								{\textbf{\subject} \hfill \name}

								\textbf{\assignment} \hfill \href{mailto:\email}{\tt \email}

								\classnum:~\semester \hfill \textbf{Due:}~\duedate~11:59 PM ET

						\hrulefill}
				\end{center}
		}{
				\begin{center}
						{\setlength{\parindent}{0cm} \setlength{\parskip}{0mm}

								{\textbf{\subject} \hfill \name\footnote{Collaborator(s): \collaborators}}

								\textbf{\assignment} \hfill \href{mailto:\email}{\tt \email}

								\classnum:~\semester \hfill \textbf{Due:}~\duedate~11:59 PM ET

						\hrulefill}
				\end{center}
		}
}

%%%%%%%%%%%%%%%%%%%%%%%%%%%%%%%%%%%
%%%%% THEOREM ENVIRONMENTS %%%%%%%%
%%%%%%%%%%%%%%%%%%%%%%%%%%%%%%%%%%%
\newtheorem{theorem}{Theorem}
\newtheorem{lemma}[theorem]{Lemma}
\newtheorem{example}[theorem]{Example}
 %% DO NOT CHANGE THIS LINE

\begin{document}
\psetheader %% DO NOT CHANGE THIS LINE

\section*{State}
Discrete-Time Markov Decision Process (MDP)

Time steps are 1 minute long

\subsection*{Environment}
\begin{itemize}
		\item Static class - no state changes (mostly)
		\item weather conditions (sunny, rainy, snowy, etc)
		\item time of day (could be encoded as a float from 0 to 24)
		\item day of week (could be encoded as an int from 0 to 6)
		\item Embedded into $l_2$
		\item Graph:
				\begin{itemize}
						\item Nodes - intersections
						\item Edges - roads
				\end{itemize}
		\item Nodes:
				\begin{itemize}
						\item Only places rides can start/end
						\item $l_2$ coordinates
						\item bool is\_charging\_station
						\item int num\_available\_chargers
				\end{itemize}
		\item Edges:
				\begin{itemize}
						\item Length
						\item 2-way speed limit, possibly asymmetric
						\item \tit{potentially:} 2-way current speed, possibly asymmetric (this would be dynamic)
				\end{itemize}
\end{itemize}
\subsection*{Fleet}
\begin{itemize}
		\item n=10? vehicles
		\item Per vehicle:
    \begin{itemize}
						\item Vehicle ID
						\item Location
						\item Battery level (percentage)
						\item Status (available, en route to pickup, with passenger, en route to charging station)
    \end{itemize}
\end{itemize}

\medskip
\subsection*{Demand}
I'm a little unsure about whether this should be \tit{explicitly} modeled in the state or
"learned" by the
agent. I'm learning towards the former for a few reasons
\begin{enumerate}
		\item It's relatively independent of the actions of the agent
		\item It can be estimated from historical data
		\item Both assumptions above are likely to be true in the real world
		\item It greatly simplifies the learning problem
\end{enumerate}
If we do explicitly model it
\begin{itemize}
		\item Location
		\item Expected demand (number of requests) in next $n$ time steps
\end{itemize}

\medskip
\subsection*{Active Rides}
I think that this class should mostly serve as an enriched tuple between a vehicle and a
ride request (described below)
\begin{itemize}
		\item Matched Vehicle ID
		\item Request ID
		\item Status (en route to pickup, with passenger)
		\item Time to pickup (minutes)
		\item Duration of ride (minutes)
		\item Estimated remaining time (minutes)
\end{itemize}

\medskip
\subsection*{Ride Requests}

There's one part of this process that I believe is non-trivial. In the real world, the
platform (e.g. Uber, Lyft) suggests a menu of prices to the rider, who then selects one.
Subsequently, all (nearby) drivers are notified of the ride request, and one can accept it.

In our simulation, if the agent recommends prices, intuitively it would be a price that the agent
would accept. So that transition probability is 1. However, you could make a case that riders
should "bid" on rides, and the agent can (1) accept (2) reject (3) counter-offer. For the sake
of simplicity, I would rather model the process as (1) customer requests a ride (2) the agent
suggests a price (3) the customer either accepts or rejects the price based on some
probabilistic model. This would allow the agent to learn to price rides optimally. It would also
provide a bound on the maximum price the agent could set (since the agent would learn to just rise
prices without bound if the customer always accepted).

In this proposed process, the price recommendation would be part of the action space, and the
customer's decision would be encoded in the transition probability.

\begin{itemize}
    % The variable $n_v$ represents a random variable following a Poisson distribution,
				% typically denoted as $n_v \sim \mathrm{Poisson}(\Lambda)$, where $\Lambda$ is the
				% rate parameter
    % (expected value) of the distribution.
		\item $n_v \sim \mathrm{Poisson}(\Lambda)$ new requests per time step (this lives in Demand class)
		\item Each request:
				\begin{itemize}
						\item Account ID (drawn from a pool of $u$ users)
						\item Request ID
						\item Pickup location
						\item Dropoff location (!= pickup location). This could be modeled as a uniform
								distribution at first, but ideally would be learned from historical data.
						\item Request time
						\item Status (categorical)
								\begin{itemize}
										\item Price suggestion pending (brand new request generated between $t$ and
												$t-1$). Agent must suggest a price before next time step
										\item Price suggested, pending customer acceptance/rejection (customer
												immediately accepts/rejects based on probabilistic model)
										\item Accepted, pending assignment to vehicle (this state only exists if there
												are no available vehicles)
										\item Assigned to vehicle, pending pickup (vehicle is en route to pickup location)
										\item Cancelled by customer (optional - if pickup time is too long)
										\item Active (joined with Vehicle class as Active Ride)
										\item Completed
								\end{itemize}
				\end{itemize}
\end{itemize}
Note that we don't need to "explicitly" model aggregate supply demand imbalance,
hopefully the agent will learn
the notion of surge pricing on its own. However, a simple float variable representing the
current ratio
of supply to demand could be added to the state to accelerate learning.

\newpage
\section*{Actions}
\subsection*{Per Vehicle}
\begin{itemize}
		\item Move (velocity vector) (always happens unless vehicle is charging)
		\item bool flip\_charging\_state (go to / from charging station)
\end{itemize}

\subsection*{Global}
\begin{itemize}
		\item $\bp = [p_1, p_2, \ldots, p_{n_v}]$ price vector for all pending requests
		\item Assign accepted requests to vehicles (if any are available)
\end{itemize}

% \newpage
\section*{Transitions}
\subsection*{Environment}
Static, no transitions (unless we want to model traffic \& speed)
\subsection*{Fleet}
\begin{itemize}
		\item Location - updated based on velocity vector, speed limits, and road network
		\item Battery level - decreases based on distance traveled, increases when at charging station
		\item Status - updated based on actions and ride assignments
\end{itemize}
\subsection*{Demand}
\begin{itemize}
		\item Updated based on historical data (e.g. time of day, day of week, weather conditions)
\end{itemize}
\subsection*{Active Rides}
\begin{itemize}
		\item Status - updated based on vehicle status and ride progress
		\item Time to pickup - decreases as vehicle approaches pickup location
		\item Duration of ride - increases as ride progresses
		\item Estimated remaining time - updated based on current time to pickup and duration of ride
\end{itemize}
\subsection*{Ride Requests}
\begin{itemize}
		\item New requests - generated based on Poisson distribution with rate parameter $\Lambda$
		\item Status - updated based on customer acceptance/rejection and ride assignments
				\begin{itemize}
						\item Expired requests - removed if not accepted within a certain time frame
						\item Cancelled requests - removed if customer cancels before assignment
						\item Completed requests - removed once ride is finished
						\item Price acceptance - probabilistic model based on suggested price and customer sensitivity
				\end{itemize}
		\item $p_{accept} = \tfrac{1}{1 + e^{-z}}$ (logistic function)
		\item z-features:
				\begin{itemize}
						\item customer bias (base acceptance rate per customer)
						\item pickup \& destination location bias (proxy for neighborhood wealth, proximity
								to public transit)
								\begin{itemize}
										\item how connected node is $\sim$ subway proximity
										\item avg income in neighborhood $\sim$ census data (or just random manifold)
								\end{itemize}
						\item price offered by agent
						\item distance
						\item global supply-demand imbalance
						\item date \& time
						\item weather conditions
				\end{itemize}
\end{itemize}

\newpage
\section*{Rewards \& Costs}
\subsection*{Rewards}
\begin{itemize}
		\item Revenue from completed rides
		\item \tit{stretch:} customer loyalty / retention
\end{itemize}
\subsection*{Costs}
\tit{Explicit}
\begin{itemize}
		\item Operational costs (electricity) - per mile cost
				\begin{itemize}
						\item Cost not paid when vehicle is charging
						\item Cost proportionate to distance (hence speed)
				\end{itemize}
		\item Customer dissatisfaction
				\begin{itemize}
						\item Rejected rides (long term effect on customer loyalty, hard to learn
								implicitly due to credit assignment problem)
						\item Long wait times
				\end{itemize}
\end{itemize}
\tit{Implicit / Learned}
\begin{itemize}
		\item Rejected ride requests (lost revenue)
		\item Unfulfilled demand (due to supply shortage; lost revenue)
		\item Long wait times (due to supply shortage | uneven vehicle distribution | traffic |
				underpricing (hence excess load))
		\item Non-shortest path routing (traffic, geography)
\end{itemize}

\newpage
\section*{Implementation Miscellany \& Thoughts}
\begin{itemize}
		\item Leverage OpenAI Gym framework
		\item Leverage existing graph libraries (e.g. NetworkX) for road network?
		\item Instant distribution of amortized costs \& rewards at each time step (speed up learning)
		\item Start incredibly simple (shrink dimensionality)
		\item Initialize with heuristic policy (e.g. nearest vehicle, price = cost * (1 + margin))
		\item Log metrics (rewards, costs, wait times, rejection rates, etc) for analysis,
				visualization, debugging \& final presentation
		\item Strict and thoughtful versioning of both code and experiments (match up version
				of code with results, reduce ambiguity in analysis)
\end{itemize}

\end{document}
