\documentclass[12pt]{article}

\newcommand{\duedate}{10/09/2025}
\newcommand{\assignment}{Architecture 10/09} % Change to "Problem Set X"

% Change the following to your name and UNI.
\newcommand{\name}{Aksel Kretsinger-Walters, adk2164}
\newcommand{\email}{adk2164@columbia.edu}

% NOTE: Defining collaborators is optional; to not list collaborators, comment out the
% line below. Maximum of two collaborators per problem set.
%\newcommand{\collaborators}{Vig Vigerton (\texttt{UNI}), Alice Bob (\texttt{UNI})}
% No collaborators on PS 0

\makeatletter
\def\input@path{{../}{../../}{../../../}} % add as many parents as you need
\makeatother
% Copyright 2021 Paolo Adajar (padajar.com, paoloadajar@mit.edu)
%
% Permission is hereby granted, free of charge, to any person obtaining a copy of this
% software and associated documentation files (the "Software"), to deal in the Software
% without restriction, including without limitation the rights to use, copy, modify,
% merge, publish, distribute, sublicense, and/or sell copies of the Software, and to
% permit persons to whom the Software is furnished to do so, subject to the following conditions:
%
% The above copyright notice and this permission notice shall be included in all copies
% or substantial portions of the Software.
%
% THE SOFTWARE IS PROVIDED "AS IS", WITHOUT WARRANTY OF ANY KIND, EXPRESS OR IMPLIED,
% INCLUDING BUT NOT LIMITED TO THE WARRANTIES OF MERCHANTABILITY, FITNESS FOR A
% PARTICULAR PURPOSE AND NONINFRINGEMENT. IN NO EVENT SHALL THE AUTHORS OR COPYRIGHT
% HOLDERS BE LIABLE FOR ANY CLAIM, DAMAGES OR OTHER LIABILITY, WHETHER IN AN ACTION OF
% CONTRACT, TORT OR OTHERWISE, ARISING FROM, OUT OF OR IN CONNECTION WITH THE SOFTWARE OR
% THE USE OR OTHER DEALINGS IN THE SOFTWARE.

%%%%%%%%%%%%%%%%%%%%%%%%%%%%%%%%%%%%%%
%%%%% DO NOT MODIFY THIS FILE %%%%%%%%
%%%%%%%%%%%%%%%%%%%%%%%%%%%%%%%%%%%%%%

%%%%%%%%%%%%%%%%%%%%%%%%%%%%%%
%%%%% CLASS SPECIFICS %%%%%%%%
%%%%%%%%%%%%%%%%%%%%%%%%%%%%%%
\newcommand{\classnum}{ORCS 4529}
\newcommand{\subject}{Reinforcement Learning}
\newcommand{\instructors}{Shipra Agrawal}
\newcommand{\semester}{Fall 2025}

%%%%%%%%%%%%%%%%%%%%%%%%%%%%%%
%%%%% PACKAGE IMPORTS %%%%%%%%
%%%%%%%%%%%%%%%%%%%%%%%%%%%%%%
\usepackage{fullpage}
\usepackage{enumitem}
\usepackage{amsfonts, amssymb, amsmath,amsthm}
\usepackage{tikz}
\usepackage{hyperref}
\usepackage{ifthen}

\hypersetup{
		colorlinks=true,
		linkcolor=blue,
		filecolor=magenta,
		urlcolor=blue,
}

%%%%%%%%%%%%%%%%%%%%%%%%%%%%
%%%%% CUSTOM MACROS %%%%%%%%
%%%%%%%%%%%%%%%%%%%%%%%%%%%%
\usepackage{macros}

%%%%%%%%%%%%%%%%%%%%%%%%%
%%%%% FORMATTING %%%%%%%%
%%%%%%%%%%%%%%%%%%%%%%%%%
\setlength{\parindent}{0mm}
\setlength{\parskip}{2mm}

\setlist[enumerate]{label=({\alph*})}
\setlist[enumerate, 2]{label=({\roman*})}

\allowdisplaybreaks[1]

%%%%%%%%%%%%%%%%%%%%%
%%%%% HEADER %%%%%%%%
%%%%%%%%%%%%%%%%%%%%%
\newcommand{\psetheader}{
		\ifthenelse{\isundefined{\collaborators}}{
				\begin{center}
						{\setlength{\parindent}{0cm} \setlength{\parskip}{0mm}

								{\textbf{\subject} \hfill \name}

								\textbf{\assignment} \hfill \href{mailto:\email}{\tt \email}

								\classnum:~\semester \hfill \textbf{Due:}~\duedate~11:59 PM ET

						\hrulefill}
				\end{center}
		}{
				\begin{center}
						{\setlength{\parindent}{0cm} \setlength{\parskip}{0mm}

								{\textbf{\subject} \hfill \name\footnote{Collaborator(s): \collaborators}}

								\textbf{\assignment} \hfill \href{mailto:\email}{\tt \email}

								\classnum:~\semester \hfill \textbf{Due:}~\duedate~11:59 PM ET

						\hrulefill}
				\end{center}
		}
}

%%%%%%%%%%%%%%%%%%%%%%%%%%%%%%%%%%%
%%%%% THEOREM ENVIRONMENTS %%%%%%%%
%%%%%%%%%%%%%%%%%%%%%%%%%%%%%%%%%%%
\newtheorem{theorem}{Theorem}
\newtheorem{lemma}[theorem]{Lemma}
\newtheorem{example}[theorem]{Example}
 %% DO NOT CHANGE THIS LINE

\begin{document}
\psetheader %% DO NOT CHANGE THIS LINE

\section*{Modeling Setup}
\subsection*{State}
Discrete-Time Markov Decision Process (MDP)

Time steps are 1 minute long

\tbf{Environment}
\begin{itemize}
		\item Static class - no state changes
		\item Embedded into $l_2$
		\item Graph -
				\begin{itemize}
						\item Nodes - intersections
						\item Edges - roads
				\end{itemize}
		\item Vertices -
				\begin{itemize}
						\item Static class - no state changes
						\item Embedded into $l_2$
				\end{itemize}
\end{itemize}
\tbf{Fleet}
\begin{itemize}
		\item n=10? vehicles
		\item Per vehicle:
    \begin{itemize}
						\item Vehicle ID
						\item Location
						\item Battery level (percentage)
						\item Status (available, en route to pickup, with passenger, en route to charging station)
    \end{itemize}
\end{itemize}

\medskip
\tbf{Demand}
I'm a little unsure about whether this should be \tit{explicitly} modeled in the state or
"learned" by the
agent. I'm learning towards the former for a few reasons
\begin{enumerate}
		\item It's relatively independent of the actions of the agent
		\item It can be estimated from historical data
		\item Both assumptions above are likely to be true in the real world
		\item It greatly simplifies the learning problem
\end{enumerate}
If we do explicitly model it
\begin{itemize}
		\item Location
		\item Expected demand (number of requests) in next time step
\end{itemize}

\medskip
\tbf{Ride Requests}
\begin{itemize}
    % The variable $n_v$ represents a random variable following a Poisson distribution,
				% typically denoted as $n_v \sim \mathrm{Poisson}(\Lambda)$, where $\Lambda$ is the
				% rate parameter
    % (expected value) of the distribution.
		\item $n_v \sim \mathrm{Poisson}(\Lambda)$ new requests per time step (this lives in Demand class)
		\item Each request:
				\begin{itemize}
						\item Request ID
						\item Pickup location
						\item Dropoff location
						\item Request time
						\item Status (pending, assigned, completed, cancelled)
				\end{itemize}
\end{itemize}

\medskip
\tbf{Active Rides}
I'm a little less sure about this one, but I think that this class should mostly serve as
an enriched
tuple between a vehicle and a ride request
\begin{itemize}
		\item Matched Vehicle ID
		\item Request ID
		\item Status (en route to pickup, with passenger)
		\item Time to pickup (minutes)
		\item Duration of ride (minutes)
		\item Estimated remaining time (minutes)
\end{itemize}
\end{document}
