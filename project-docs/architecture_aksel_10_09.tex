\documentclass[12pt]{article}

\newcommand{\duedate}{10/09/2025}
\newcommand{\assignment}{Architecture 10/09} % Change to "Problem Set X"

% Change the following to your name and UNI.
\newcommand{\name}{Aksel Kretsinger-Walters, adk2164}
\newcommand{\email}{adk2164@columbia.edu}

% NOTE: Defining collaborators is optional; to not list collaborators, comment out the
% line below. Maximum of two collaborators per problem set.
%\newcommand{\collaborators}{Vig Vigerton (\texttt{UNI}), Alice Bob (\texttt{UNI})}
% No collaborators on PS 0

\makeatletter
\def\input@path{{../}{../../}{../../../}} % add as many parents as you need
\makeatother
% Copyright 2021 Paolo Adajar (padajar.com, paoloadajar@mit.edu)
%
% Permission is hereby granted, free of charge, to any person obtaining a copy of this
% software and associated documentation files (the "Software"), to deal in the Software
% without restriction, including without limitation the rights to use, copy, modify,
% merge, publish, distribute, sublicense, and/or sell copies of the Software, and to
% permit persons to whom the Software is furnished to do so, subject to the following conditions:
%
% The above copyright notice and this permission notice shall be included in all copies
% or substantial portions of the Software.
%
% THE SOFTWARE IS PROVIDED "AS IS", WITHOUT WARRANTY OF ANY KIND, EXPRESS OR IMPLIED,
% INCLUDING BUT NOT LIMITED TO THE WARRANTIES OF MERCHANTABILITY, FITNESS FOR A
% PARTICULAR PURPOSE AND NONINFRINGEMENT. IN NO EVENT SHALL THE AUTHORS OR COPYRIGHT
% HOLDERS BE LIABLE FOR ANY CLAIM, DAMAGES OR OTHER LIABILITY, WHETHER IN AN ACTION OF
% CONTRACT, TORT OR OTHERWISE, ARISING FROM, OUT OF OR IN CONNECTION WITH THE SOFTWARE OR
% THE USE OR OTHER DEALINGS IN THE SOFTWARE.

%%%%%%%%%%%%%%%%%%%%%%%%%%%%%%%%%%%%%%
%%%%% DO NOT MODIFY THIS FILE %%%%%%%%
%%%%%%%%%%%%%%%%%%%%%%%%%%%%%%%%%%%%%%

%%%%%%%%%%%%%%%%%%%%%%%%%%%%%%
%%%%% CLASS SPECIFICS %%%%%%%%
%%%%%%%%%%%%%%%%%%%%%%%%%%%%%%
\newcommand{\classnum}{ORCS 4529}
\newcommand{\subject}{Reinforcement Learning}
\newcommand{\instructors}{Shipra Agrawal}
\newcommand{\semester}{Fall 2025}

%%%%%%%%%%%%%%%%%%%%%%%%%%%%%%
%%%%% PACKAGE IMPORTS %%%%%%%%
%%%%%%%%%%%%%%%%%%%%%%%%%%%%%%
\usepackage{fullpage}
\usepackage{enumitem}
\usepackage{amsfonts, amssymb, amsmath,amsthm}
\usepackage{tikz}
\usepackage{hyperref}
\usepackage{ifthen}

\hypersetup{
		colorlinks=true,
		linkcolor=blue,
		filecolor=magenta,
		urlcolor=blue,
}

%%%%%%%%%%%%%%%%%%%%%%%%%%%%
%%%%% CUSTOM MACROS %%%%%%%%
%%%%%%%%%%%%%%%%%%%%%%%%%%%%
\usepackage{macros}

%%%%%%%%%%%%%%%%%%%%%%%%%
%%%%% FORMATTING %%%%%%%%
%%%%%%%%%%%%%%%%%%%%%%%%%
\setlength{\parindent}{0mm}
\setlength{\parskip}{2mm}

\setlist[enumerate]{label=({\alph*})}
\setlist[enumerate, 2]{label=({\roman*})}

\allowdisplaybreaks[1]

%%%%%%%%%%%%%%%%%%%%%
%%%%% HEADER %%%%%%%%
%%%%%%%%%%%%%%%%%%%%%
\newcommand{\psetheader}{
		\ifthenelse{\isundefined{\collaborators}}{
				\begin{center}
						{\setlength{\parindent}{0cm} \setlength{\parskip}{0mm}

								{\textbf{\subject} \hfill \name}

								\textbf{\assignment} \hfill \href{mailto:\email}{\tt \email}

								\classnum:~\semester \hfill \textbf{Due:}~\duedate~11:59 PM ET

						\hrulefill}
				\end{center}
		}{
				\begin{center}
						{\setlength{\parindent}{0cm} \setlength{\parskip}{0mm}

								{\textbf{\subject} \hfill \name\footnote{Collaborator(s): \collaborators}}

								\textbf{\assignment} \hfill \href{mailto:\email}{\tt \email}

								\classnum:~\semester \hfill \textbf{Due:}~\duedate~11:59 PM ET

						\hrulefill}
				\end{center}
		}
}

%%%%%%%%%%%%%%%%%%%%%%%%%%%%%%%%%%%
%%%%% THEOREM ENVIRONMENTS %%%%%%%%
%%%%%%%%%%%%%%%%%%%%%%%%%%%%%%%%%%%
\newtheorem{theorem}{Theorem}
\newtheorem{lemma}[theorem]{Lemma}
\newtheorem{example}[theorem]{Example}
 %% DO NOT CHANGE THIS LINE

\begin{document}
\psetheader %% DO NOT CHANGE THIS LINE

\section*{Modeling Setup}
\subsection*{State}
Discrete-Time Markov Decision Process (MDP)

Time steps are 1 minute long

\tbf{Environment}
\begin{itemize}
		\item Locations will be represented either/or as latitude/longitude coordinates or as
				nodes on a graph (any point on vertex is proportional distance between endpoints)
\end{itemize}
\tbf{Fleet}
\begin{itemize}
		\item n=10? vehicles
		\item Per vehicle:
    \begin{itemize}
						\item Vehicle ID
						\item Location
						\item Battery level (percentage)
						\item Status (available, en route to pickup, with passenger, en route to charging station)
    \end{itemize}
\end{itemize}

\tbf{Demand}
\begin{itemize}
		\item
\end{itemize}
\tbf{Ride Requests}
\begin{itemize}
    % The variable $n_v$ represents a random variable following a Poisson distribution,
				% typically denoted as $n_v \sim \mathrm{Poisson}(\Lambda)$, where $\Lambda$ is the
				% rate parameter
    % (expected value) of the distribution.
		\item $n_v \sim \mathrm{Poisson}(\Lambda)$ new requests per time step
		\item Each request:
				\begin{itemize}
						\item Request ID
						\item Pickup location
						\item Dropoff location
						\item Request time
						\item Status (pending, assigned, completed, cancelled)
				\end{itemize}
\end{itemize}
\tbf{Active Rides}
I'm a little less sure about this one, but I think that this class should mostly serve as
an enriched
tuple between a vehicle and a ride request
\begin{itemize}
		\item Matched Vehicle ID
		\item Request ID
		\item Status (en route to pickup, with passenger)
		\item Time to pickup (minutes)
		\item Duration of ride (minutes)
		\item Estimated remaining time (minutes)
\end{itemize}
\newpage
We model the state as a tuple $(t, F, R, D)$ where:
\begin{itemize}
		\item $t$ is the current time step (discretized to 10 minute intervals)
		\item $F$ is a list of all vehicles in the fleet, where each vehicle is represented as
				a tuple $(id, loc, status, battery, time\_to\_maintenance)$
    \begin{itemize}
						\item $id$ is a unique identifier for the vehicle
						\item $loc$ is the current location of the vehicle (latitude, longitude)
						\item $status$ is the current status of the vehicle (available, en route to pickup,
								with passenger, en route to charging station, en route to maintenance)
						\item $battery$ is the current battery level of the vehicle (percentage)
						\item $time\_   to\_maintenance$ is the time remaining until the vehicle requires
								maintenance (in hours)
    \end{itemize}
		\item $R$ is a list of all ride requests, where each request is represented as a tuple
				$(id, pickup\_loc, dropoff\_loc, request\_time, status)$
    \begin{itemize}
						\item $id$ is a unique identifier for the request
						\item $pickup\_loc$ is the location of the pickup (latitude, longitude)
						\item $dropoff\_loc$ is the location of the dropoff (latitude, longitude)
						\item $request\_time$ is the time the request was made
						\item $status$ is the current status of the request (pending, assigned, completed, cancelled)
    \end{itemize}
		\item $D$ is a list of all demand hotspots, where each hotspot is represented as a
				tuple $(loc, expected\_demand)$
    \begin{itemize}
						\item $loc$ is the location of the hotspot (latitude, longitude)
						\item $expected\_demand$ is the expected number of ride requests in the next time step
    \end{itemize}
\end{itemize}
\end{document}
