\documentclass[12pt]{article}

% \newcommand{\duedate}{10/09/2025}
% \newcommand{\assignment}{RL Project Proposal} % Change to "Problem Set X"

% % Change the following to your name and UNI.
% \newcommand{\name}{Aksel Kretsinger-Walters, adk2164}
% \newcommand{\email}{adk2164@columbia.edu}

% % NOTE: Defining collaborators is optional; to not list collaborators, comment out the
% % line below. Maximum of two collaborators per problem set.
\newcommand{\collaborators}{Aksel Kretsinger-Walters (\texttt{adk2164}), Alena Chan
(\texttt{ac5477}), Andrey Aksyutkin (\texttt{aa5499}), Blake Sisson (\texttt{mbs2246})}
% No collaborators on PS 0

\makeatletter
\def\input@path{{../}{../../}{../../../}} % add as many parents as you need
\makeatother
% Copyright 2021 Paolo Adajar (padajar.com, paoloadajar@mit.edu)
%
% Permission is hereby granted, free of charge, to any person obtaining a copy of this
% software and associated documentation files (the "Software"), to deal in the Software
% without restriction, including without limitation the rights to use, copy, modify,
% merge, publish, distribute, sublicense, and/or sell copies of the Software, and to
% permit persons to whom the Software is furnished to do so, subject to the following conditions:
%
% The above copyright notice and this permission notice shall be included in all copies
% or substantial portions of the Software.
%
% THE SOFTWARE IS PROVIDED "AS IS", WITHOUT WARRANTY OF ANY KIND, EXPRESS OR IMPLIED,
% INCLUDING BUT NOT LIMITED TO THE WARRANTIES OF MERCHANTABILITY, FITNESS FOR A
% PARTICULAR PURPOSE AND NONINFRINGEMENT. IN NO EVENT SHALL THE AUTHORS OR COPYRIGHT
% HOLDERS BE LIABLE FOR ANY CLAIM, DAMAGES OR OTHER LIABILITY, WHETHER IN AN ACTION OF
% CONTRACT, TORT OR OTHERWISE, ARISING FROM, OUT OF OR IN CONNECTION WITH THE SOFTWARE OR
% THE USE OR OTHER DEALINGS IN THE SOFTWARE.

%%%%%%%%%%%%%%%%%%%%%%%%%%%%%%%%%%%%%%
%%%%% DO NOT MODIFY THIS FILE %%%%%%%%
%%%%%%%%%%%%%%%%%%%%%%%%%%%%%%%%%%%%%%

%%%%%%%%%%%%%%%%%%%%%%%%%%%%%%
%%%%% CLASS SPECIFICS %%%%%%%%
%%%%%%%%%%%%%%%%%%%%%%%%%%%%%%
\newcommand{\classnum}{ORCS 4529}
\newcommand{\subject}{Reinforcement Learning}
\newcommand{\instructors}{Shipra Agrawal}
\newcommand{\semester}{Fall 2025}

%%%%%%%%%%%%%%%%%%%%%%%%%%%%%%
%%%%% PACKAGE IMPORTS %%%%%%%%
%%%%%%%%%%%%%%%%%%%%%%%%%%%%%%
\usepackage{fullpage}
\usepackage{enumitem}
\usepackage{amsfonts, amssymb, amsmath,amsthm}
\usepackage{tikz}
\usepackage{hyperref}
\usepackage{ifthen}

\hypersetup{
		colorlinks=true,
		linkcolor=blue,
		filecolor=magenta,
		urlcolor=blue,
}

%%%%%%%%%%%%%%%%%%%%%%%%%%%%
%%%%% CUSTOM MACROS %%%%%%%%
%%%%%%%%%%%%%%%%%%%%%%%%%%%%
\usepackage{macros}

%%%%%%%%%%%%%%%%%%%%%%%%%
%%%%% FORMATTING %%%%%%%%
%%%%%%%%%%%%%%%%%%%%%%%%%
\setlength{\parindent}{0mm}
\setlength{\parskip}{2mm}

\setlist[enumerate]{label=({\alph*})}
\setlist[enumerate, 2]{label=({\roman*})}

\allowdisplaybreaks[1]

%%%%%%%%%%%%%%%%%%%%%
%%%%% HEADER %%%%%%%%
%%%%%%%%%%%%%%%%%%%%%
\newcommand{\psetheader}{
		\ifthenelse{\isundefined{\collaborators}}{
				\begin{center}
						{\setlength{\parindent}{0cm} \setlength{\parskip}{0mm}

								{\textbf{\subject} \hfill \name}

								\textbf{\assignment} \hfill \href{mailto:\email}{\tt \email}

								\classnum:~\semester \hfill \textbf{Due:}~\duedate~11:59 PM ET

						\hrulefill}
				\end{center}
		}{
				\begin{center}
						{\setlength{\parindent}{0cm} \setlength{\parskip}{0mm}

								{\textbf{\subject} \hfill \name\footnote{Collaborator(s): \collaborators}}

								\textbf{\assignment} \hfill \href{mailto:\email}{\tt \email}

								\classnum:~\semester \hfill \textbf{Due:}~\duedate~11:59 PM ET

						\hrulefill}
				\end{center}
		}
}

%%%%%%%%%%%%%%%%%%%%%%%%%%%%%%%%%%%
%%%%% THEOREM ENVIRONMENTS %%%%%%%%
%%%%%%%%%%%%%%%%%%%%%%%%%%%%%%%%%%%
\newtheorem{theorem}{Theorem}
\newtheorem{lemma}[theorem]{Lemma}
\newtheorem{example}[theorem]{Example}
 %% DO NOT CHANGE THIS LINE

% Override the template margins to be less than 1 inch
% Load geometry after the template so it takes precedence.
\usepackage[margin=0.5in]{geometry}

\title{Waymo Fleet Profitability Optimizer}
\author{
		Aksel Kretsinger-Walters \and
		Alena Chan \and
		Andrey Aksyutkin \and
		Blake Sisson \footnote{Collaborator(s): \collaborators}
}
\date{}
\begin{document}
\maketitle

\subsection*{Problem Statement}
One of the most promising and revolutionary applications of reinforcement learning is in
the domain of autonomous robots, specifically self-driving cars. There are many challenges
in this domain: intellectual, ethical, technical, and more. For our project, we've decided
to narrow our focus to the specific problem of optimizing the profitability of a fleet of
self-driving cars.

Monitoring, maintaining, and optimizing a large fleet of self driving cars is a complex
problem, and one can quickly think of many dimensions that the problem takes on.
Predicting demand, scheduling maintenance, recharging vehicles, setting competitive
prices, maximizing coverage, minimizing wait times, and more are all separately non-trivial
problems. Jointly optimizing across all of these dimensions and adapting to distribution shifts
is an even more challenging problem, and the interactive nature of the problem makes it a natural
fit for reinforcement learning and agentic approaches.

We plan to simulate a fleet of self-driving cars as a Markov Decision Process (MDP), and
develop reinforcement learning algorithms to optimize the fleet's operations and profitability.

\subsection*{Interest and Relevance}
This is an area of active research and development, with many companies investing heavily in
self-driving technology. For this technology to reach the market, it is absolutely
critical to resolve the safety and reliability challenges that currently exist; however, the
economic viability must also be solved for self-driving to reach its full potential.

For the purposes of this course, we think that the interactive nature of the state with its
ecosystem, the high dimensionality of the state and action spaces, the data and research publicly
available and the possibility of performing far better than a heuristic algorithm make the
project a good fit for our semester project.

\begin{thebibliography}{9}

		\bibitem{lehd} U.S. Census Bureau. "LEHD Origin-Destination Employment Statistics
		(LODES)." \url{https://lehd.ces.census.gov/data/}
		\bibitem{suttonbarto} Sutton, R. S., \& Barto, A. G. "Reinforcement Learning: An
		Introduction." MIT Press, 2018.
		\bibitem{waymoopen} Waymo. "Waymo Open Dataset." \url{https://waymo.com/open/}
		\bibitem{citibike_rl} Xiao, I. "Reinforcement Learning Project: CitiBike."
		\url{https://github.com/ianxxiao/reinforcement_learning_project/blob/master/Reports/Presentation_RL_citiBike_20180514.pdf}
		\bibitem{ridesharing_survey} Li, J., Li, X., \& Wang, F. "Reinforcement Learning for
		Ridesharing: An Extended Survey." arXiv preprint arXiv:2102.11896, 2021.
\end{thebibliography}
\end{document}
